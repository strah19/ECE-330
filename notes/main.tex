\documentclass[12pt]{article}

\title{UW Madison E C E 330 Notes}
\author{Strahinja Marinkovic}
\date{\today}

\begin{document}

\maketitle

\section*{Notes}

\begin{itemize}
    \item Considering what the prerequisites are for this course, it might be a good idea to have a video or a document/worksheet for the needed math. Or each required concept could just be clearly stated/mentioned before a new module (i.e. partial fractions for laplace).
    \item We should introduce applied homeworks. This is something similar to what E C E 352 does. We should find a project that each week we write code that builds up to some final project that uses all the work from the semester. Could be something like implementing sampling/reconstructing with some sort of DSP where each week we make some part of the larger project.
    \item How difficult is it to change a course description? Can we add something where MATH 320 or MATH 340 is highly recommended but not required?
    \item I think that exercises during class should almost guide you through the content a little more. I feel like it should be more than just a short question with a single text box answer. I know some questions are more involved and those are the ones where students are more likely to make connections and develop intuition. 
    \item What's the plan for differential equations? Are we going to stick with how the course does it now with steady state and transient response or go with how the book does it with zero input response, etc.
\end{itemize}

\newpage
\section*{Student Comments/Observations}

\textbf{These comments and observations come from tutoring, they are not necessarily generally true and come from specific students.}

\begin{itemize}
    \item Time invariance could be taught better with more examples. More graphical intuition.
    \item Stability was not made clear considering its the first question on the exam.
    \item I've noticed for Fourier students rely too heavily on the formula sheet. For me, it should be used an important reference and tool but it's not the first place I go to solve a problem. I've had students immediately go to the formula sheet after reading a question without really thinking about whats being asked and over think. It shows me a lack of intuition on what Fourier actually is.
    \item There was a question a student was stuck on where it was a Fourier Transform of a sum of impulses. The question wanted you to use sifting property after using the integral definition of the Fourier Transform. I like this question, its a good combination of two concepts, however, what would the learning outcome be? \textbf{I feel like students don't know what they are supposed to know, which makes it hard to study for.} "Students should be able to utilize signal properties to solve Fourier Transforms"?
    \item Students were confused about the precedence of Fourier Transform properties. For example, "should I do a time shift or a frequency shift first?", "When exactly is linearity applied?"
\end{itemize}

\end{document}
